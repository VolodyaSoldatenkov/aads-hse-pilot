\documentclass[a4paper,12pt]{article}
\usepackage{cmap}
\usepackage[T2A]{fontenc}
\usepackage[utf8]{inputenc}
\usepackage[english,russian]{babel}
\usepackage{mathtext}
\usepackage{amsmath,amsfonts,amssymb,amsthm,mathtools}
\usepackage{icomma}
\usepackage{listings}

\lstset{language=C++}

\title{Случайные величины и мат.ожидание}
\date{}

\begin{document}
	\maketitle
	\noindent $\xi:  \Omega \rightarrow \mathbb{R}$ - \textbf{случайная величина} (численное выражение события)

	\noindent Пример: $p(\xi=3)=\sum\limits_{\omega: \xi(\omega)=3} p(\omega)$\\

	\noindent Пусть  $\Omega \rightarrow \xi_1$, $\Omega \rightarrow \xi_2$ - случайные величины.

	\noindent Они \textbf{независимы}, если $\forall A, B \subset \mathbb{R}$ независимы $\xi_1 \subset A$ и $\xi_1 \subset B$.\\

	\noindent $A$ - событие. Тогда \textbf{индикаторная случайная величина} определяется следующим образом:
	\begin{equation*}
	I_a(\omega) =
	\begin{cases}
	1, \; \omega\in A\\
	0, \; \omega \in A
	\end{cases}
	\end{equation*}\\

	\noindent $E(\xi)=\sum\limits_{\omega \in \Omega} \xi(\omega)p(w)$ - \textbf{мат.ожидание} (среднее значение случайной величины) \\

	\subsection*{Свойства мат.ожидания}
	$E(c\xi)=cE(\xi)$\\


   	\noindent $E(\alpha\xi_1+\beta\xi_2)=\alpha E(\xi_1)+\beta E(\xi_2)$
	\begin{proof}
	По определению:
	$E(\alpha\xi_1+\beta\xi_2)=\sum\limits_{\omega \in \Omega}(\alpha\xi_1(\omega)+\beta\xi_2(\omega))=\alpha E(\xi_1)+\beta E(\xi_2) $
	\end{proof}
    ~\

    \newpage
	\noindent $E(\xi_1\xi_2)=E(\xi_1)E(\xi_2)$, если $\xi_1, \xi_2$ независимы.
	\begin{proof}
	~\

	\noindent $E(\xi)=\sum\limits_{x \in \mathbb{R}}x\cdot p(\xi=x)$\\
	\noindent $E(\xi_1\xi_2)=\sum\limits_{x \in \mathbb{R}}x\cdot p(\xi_1\xi_2=x)=\sum\limits_{y \in \mathbb{R}}\sum\limits_{z \in \mathbb{R}}y z \cdot p(y \cap z)=\sum\limits_{y \in \mathbb{R}}\sum\limits_{z \in \mathbb{R}}y z \cdot p(y)p(z)=\sum\limits_{y \in \mathbb{R}}p(\xi_1=y)+\sum\limits_{z \in \mathbb{R}}p(\xi_2=z)$
	\end{proof}

	\noindent $D(\xi)=\sum\limits_{\omega \in \Omega}(\xi(\omega-E(\xi)))^2p(\omega)=E(\xi-E(\xi))$ - \textbf{дисперсия} (разброс значений)

	\subsection*{Примеры}
	1. Есть $n$ рабочих стоимостью $c_1, c_2, \cdots, c_n$ за единицу времени, и $n$ работ занимающие $t_1, t_2, \cdots, t_n$ времени. Сколько ожидаемо мы заплатим случайной перестановке рабочих? \\
	$E(\xi \rho)=E\sum\limits_{i=1}^n c(\rho_i)t_i=\sum\limits_{i=1}^n t_i E(c_{\rho_i})=\sum\limits_{i=1}^n t_i \frac{\sum c_i}{n}=\frac{\sum t_i\sum c_i}{n}$ (из линейности мат.ожидания, раскладываем на $n$ функций) \\
	~\

	\noindent 2. Пусть задан граф $G=(V,E)$, и $A \subset V$. Тогда разрез $U_A$ - это все ребра, соединяющие вершины из А с вершинами не из А. Каково мат.ожидание его размера?\\
	$E|U_A|=E\sum\limits_{e \in E} I_{e \in U_A}=\sum\limits_{e \in E}p(e \in U_A)=\frac{|E|}{2}$ (т.к. каждая вершина относится к $A$ с вероятностью $\frac{1}{2}$)\\
	~\

	\noindent 3. Каково мат.ожидание $len$ в жадном поиске НВП?\\
	\begin{lstlisting}
	lst=-1
	len=0
	for i: 1..n
	    if p[i] > lst:
	    	lst = p[i]
	    	len += 1
	\end{lstlisting}

	\noindent $E(\xi(\rho))=E\sum\limits_{i=1}^n I_{\rho_i=max(\rho_1, \cdots, \rho_i)}=\sum\limits_{i=1}^n \frac{1}{i}$ (т.к. вероятность, что максимум в
	последовательности из $i$ элементов будет на $i$-м месте, равна $\frac{1}{i}$)

	\subsection*{Оценка дисперсии}
	\textbf{Неравенство Маркова}\\
	$p(\xi>kE(\xi)) \le \frac{1}{k}$ при $\xi(w) \ge 0$, $E(\xi) \ge 0$\\

	\noindent \textbf{Неравенство Чебышева}\\
	\noindent $p(|\xi-E(\xi)|>a) \le \frac{D(\xi)}{a^2}$

\end{document}
