\documentclass[a4paper, 12pt]{article}

\usepackage{packages}

\title{АиСД, пилотный поток. Лекция 3.}
\author{}
\date{}

\begin{document}
    \maketitle

    Немного философии. Последовательность действий.
    \begin{enumerate}
        \item Формулировка задачи;
        \item Выработка алгортима;
        \item Доказательство корректности и асимптотики;
        \item \(\Omega\)-оценка алгоритма;
        \item \(\Omega\)-оценка задачи.
    \end{enumerate}

    Модель вычислительной машины, какой мы её видим:
    \begin{itemize}
        \item Программа отделена от памяти;
        \item Память линейна, состоит из ячеек, которые можно адресовать;
        \item В каждой ячейке лежит число;
        \item Выделить память можно только за линейное время от её размера;
        \item Нам доступны арифметика, условная и безусловная передача, адресация, отношения, битовые операции и (возможно) математические функции.
    \end{itemize}

    Способы доказательства корректности алгоритмов:
    \begin{enumerate}
        \item От противного;
        \item По индукции;
        \item По инварианту.
    \end{enumerate}

    Способы доказательства асимптотики алгоритмов:
    \begin{enumerate}
        \item Метод прямого учёта;
        \item Рекурренты;

    \end{enumerate}
\end{document}
